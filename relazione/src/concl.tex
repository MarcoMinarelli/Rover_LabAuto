\label{sec:concl}

\subsection{Sviluppi Futuri}
  Possibili sviluppi futuri: 
  \begin{itemize}
    \item Implementazione filtro di Kalman (Correzione - Predizione) per una migliore stima della velocità lineare
    \item Implementazione Firmware per poter utilizzare gli encoder presenti nei motori di sterzo e trazione
    \item Aggiungere un router Wi-Fi a bordo del rover così da semplificare la connessione
    \item Integrazione di una memoria SSD con maggiore capacità
  \end{itemize}
  
\subsection{Conclusioni}
  Gli obiettivi fissati sono stati raggiunti come dimostrato dai grafici, nonostante gli errori sulla posa introdotti dalla ZED. Il nodo di visione riesce correttamente ad individuare i marker e gli id ad essi associati, il nodo di controllo sceglie correttamente il punto a cui convergere tra i due proposti  e riesce a svoltare in modo smooth.
  Per quanto riguarda l'obstacle avoidance il sistema evita gli ostacoli la cui posizione è conosciuta a priori dimostrando la bontà dell'implementazione di APF, mentre non è stato possibile testare APF nel caso di ostacoli dinamici a causa della mancata comprensione del funzionamento del LiDAR. Il nodo convertitore, tramite l'utilizzo di due PID correttamente regolati, traduce con successo le grandezze solitamente utilizzate nel controllo dei veicoli ($v$ e $\omega$) nelle grandezze impiegate nel dart\_wrapper. 
