\label{sec:impl}
\subsection{Nodo di visione}
Gli obiettivi del nodo visione sono:
\begin{enumerate}
  \item riconoscere nell'immagine i markers ArUco;
  \item localizzarli nella mappa;
  \item ricavare l'id del marker e quindi il coefficiente angolare della retta a cui il robot dovrà convergere.
\end{enumerate}
  Per quanto riguarda il primo punto, la libreria OpenCV \cite{opencv_library} mette a disposizione un modulo che permette di ricavare la posizione degli angoli del 
  marker e di ricavare l'id associato. \\
  Riguardo il secondo punto, supponendo di conoscere la matrice di calibrazione $K$ della camera \cite{multiple_view} e la posizione dell'ArUco in coordinate omogenee 
  $p_w = (X \: Y \: Z \: 1 )^T$, se definiamo il punto immagine come in coordinate pixel $ p_c =  (u \: v \: 1)^T$ si ottiene 
    \begin{equation}
    p_c \propto K \Pi [R | \boldsymbol{t}] p_w  \Rightarrow
     \begin{bmatrix}
           u \\
           v \\
           1
        \end{bmatrix}
         \propto 
         \begin{bmatrix}
           f_x & \sigma & c_x \\
           0 & f_y & c_y \\
           0 & 0 & 1 
         \end{bmatrix}
          \begin{bmatrix}
           1 & 0 & 0 & 0 \\
           0 & 1 & 0 & 0 \\
           0 & 0 & 1 & 0
         \end{bmatrix}
         \begin{bmatrix}
           r_{11} & r_{12} & r_{13} & t_1 \\
           r_{21} & r_{22} & r_{23} & t_2 \\
           r_{31} & r_{32} & r_{33} & t_3  \\
           0 & 0 & 0 & 1
         \end{bmatrix}
         \begin{bmatrix}
           X \\
           Y \\
           Z \\
           1
    \end{bmatrix}
  \end{equation}
  con $R$ matrice di rotazione e $ \boldsymbol{t}$ vettore che indica la traslazione tra la camera e il marker; calcolando questi due elementi si può ricavare la 
  posa del target in terna camera. Per risolvere questo problema sono stati realizzati vari algoritmi \cite{marchand2015pose}, i quali possono essere richiamati dalla 
  funzione \texttt{solve\_PnP()} della libreria. Nello specifico in \cite{infinitesimal} è proposto un metodo per risolvere il problema nel caso il target siano dei
  marker. \\
  La matrice di trasformazione dalla terna Robot alla terna camera è data da \footnote{In realtà la matrice di trasformazione dovrebbe essere $T^R_C = \begin{bmatrix} 0 & 0 & 1 & 0 \\ -1 & 0 & 0 & 0 \\ 0 & -1 & 0 & 0 \\ 0 & 0 & 0 & 1 \end{bmatrix} $. Ma, dato che la camera è ruotata di $180^{\circ}$ si ottiene la matrice sopra riportata. }   
\begin{equation}
T^R_C = \begin{bmatrix} 0 & 0 & 1 & 0 \\ 1 & 0 & 0 & 0 \\ 0 & 1 & 0 & 0 \\ 0 & 0 & 0 & 1 \end{bmatrix} 
\end{equation}\\ mentre la matrice di trasformazione dalla terna camera alla terna marker da 
\begin{equation}
T^C_M = \begin{bmatrix} R & \boldsymbol{t} \\ \begin{matrix} 0 & 0  & 0 \end{matrix} & 1 \end{bmatrix}
\end{equation}
  A partire quindi dalla posa del marker rispetto a camera e dalla posa della camera in terna fissa si può ottenere la posizione in terna fissa del marker $(x_A, y_A)$.\\

  L'id del marker identifica sia i parametri della retta che la direzione desiderata come espresso in 
  \begin{table} [H]
    \centering
    \begin{tabular}{|ccc|}
    \hline
        Id marker & Equazione retta & Direzione \\  \hline
        0 &  $ y = y_A$  & Destra   \\  \hline
        1 &  $ x= x_A $  & Destra\\  \hline
        2 &  $ y = y_A $ & Sinistra\\  \hline
        3 &  $ x = x_A $ & Sinistra\\  \hline
    \end{tabular}
\end{table}
  
\subsection{Nodo di controllo}
\subsubsection{APF}
Per poter utilizzare tale tipologia di controllo bisogna calcolare il \textit{goal} dinamicamente, cioè calcolare l'intersezione tra la retta e il cerchio di raggio $r$. Ricordiamo che le coordinate $(x_r,y_r)$ coincidono con quelle del rover. \\
Si possono distinguere due casi, quello in cui la retta è definita come $y=mx+q$ e quello in cui si ha $x=cost$.\\
Riferendoci al primo caso, si imposta il sistema:
\begin{equation} 
\begin{cases}

    (x-x_r)^2+(y-y_r)^2=r^2
   \\
    y=mx+q 
  \end{cases} 
\end{equation}
si sostituisce la seconda equazione nella prima
\begin{equation*}
(x-x_r)^2+(mx+q-y_r)^2=r^2
\end{equation*}

\begin{equation}
\Rightarrow x^2+x_r^2-2xx_r+m^2x^2+q^2+y_r^2+2mxq-2mqy_r-2mxy_r-r^2=0
\end{equation}

\begin{equation*}
\Rightarrow \underbrace{(1+m^2)}_\text{a}x^2+\underbrace{(2mq-2x_r-2my_r)}_\text{b}x+\underbrace{(x_r^2+q^2+y_r^2-2qy_r-r^2)}_\text{c}=0
\end{equation*}
Definiamo $\Delta=b^2-4ac=(2mq-2x_r-2my_r)^2-4(1+m^2)(x_r^2+q^2+y_r^2-2qy_r-r^2)$. \\Se:
\begin{itemize}
    \item $\Delta>0$ si hanno due soluzioni
        \begin{equation}
        x_1=\frac{-(2mq-2x_r-2my_r)+\sqrt{2mq-2x_r-2my_r)^2-4(1+m^2)(x_r^2+q^2+y_r^2-2qy_r-r^2)}}{2(1+m^2)}
        \end{equation}
        \\
        \begin{equation}
        x_2=\frac{-(2mq-2x_r-2my_r)-\sqrt{2mq-2x_r-2my_r)^2-4(1+m^2)(x_r^2+q^2+y_r^2-2qy_r-r^2)}}{2(1+m^2)}
        \end{equation}
        Se la distanza risulta maggiore del raggio $r$, si sceglie la soluzione più lontana rispetto alla posizione dell'ArUco.
        Nel caso in cui invece la distanza risulti minore del raggio si sceglie 
        \begin{itemize}
            \item quella con il coefficiente angolare $m$ minore nel caso in cui il rover giri a destra
            \item quella con il coefficicnte angolare maggiore nel caso in cui il river giri a sinistra
        \end{itemize}
    \item $\Delta=0$ si ha una sola soluzione
    \begin{equation}
        x_d=\frac{-(2mq-2x_r-2my_r)}{2(1+m^2)}
        \end{equation}
    \item $\Delta<0$ non si hanno soluzioni, per cui usiamo la proiezione ortogonale tra la retta $y=mx+q$ e la retta passante per le coordinate $(x_r, y_r)$ del rover.
\end{itemize}
Riferendoci al secondo caso invece, si imposta il sistema:

\begin{equation} 
\begin{cases}

    (x-x_r)^2+(y-y_r)^2=r^2
   \\
    x=x_a 
  \end{cases} 
\end{equation}
si sostituisce la seconda equazione nella prima
\begin{equation*}
(x_a-x_r)^2+(y-y_r)^2=r^2
\end{equation*}

\begin{equation}
\Rightarrow x_a^2+x_r^2-2x_ax_r+y^2+y_r^2-2yy_r-r^2=0
\end{equation}

\begin{equation*}
\Rightarrow y^2+\underbrace{(-2y_r)}_\text{b}y+\underbrace{(x_r^2+x_a^2-2x_ax_r+y_r^2-r^2)}_\text{c}=0
\end{equation*}
Definiamo $\Delta=b^2-4ac=(-2y_r)^2-4(x_r^2+x_a^2-2x_ax_r+y_r^2-r^2)$. \\Se:
\begin{itemize}
    \item $\Delta>0$ si hanno due soluzioni
        \begin{equation}
        y_1=\frac{(2y_r)+\sqrt{(-2y_r)^2-4(x_r^2+x_a^2-2x_ax_r+y_r^2-r^2)}}{2}
        \end{equation}
        \\
        \begin{equation}
        y_2=\frac{(2y_r)-\sqrt{(-2y_r)^2-4(x_r^2+x_a^2-2x_ax_r+y_r^2-r^2)}}{2}
        \end{equation}
  Se la distanza risulta maggiore del raggio $r$, si sceglie la soluzione più lontana rispetto alla posizione dell'ArUco.
        Nel caso in cui invece la distanza risulti minore del raggio si sceglie 
        \begin{itemize}
            \item quella con il coefficiente angolare $m$ minore nel caso in cui il rover giri a destra
            \item quella con il coefficicnte angolare maggiore nel caso in cui il river giri a sinistra
        \end{itemize}    \item $\Delta=0$ si ha una sola soluzione
    \begin{equation}
        y_d=y_r
        \end{equation}
    \item $\Delta<0$ non si hanno soluzioni, per cui usiamo la proiezione ortogonale tra la retta $x=cost$ e la retta $y=cost$ pasante per $y_r$.
\end{itemize}
Il punto così ottenuto è il $\boldsymbol{q_G}$ della \autoref{APF}.
Si applicano quindi le formule viste in tale sottosezione con i seguenti parametri:

\begin{table} [H]
    \centering
    \begin{tabular}{|cc|}
    \hline
        Parametro & Valore usato \\  \hline
        $k_a$ &   3   \\  \hline
        $k_r$ &   3\\  \hline
        $\delta$ & 0.4  m \\  \hline
        $\delta_0$ &  1  m\\  \hline
    \end{tabular}
\end{table}

\subsection{Nodo convertitore}
Tale nodo è stato implementetato con lo scopo di ottenere la velocità angolare e lineare a partire dall'accelerazione e dallo sterzo. Dopo aver ricavato tali parametri questi vengono inviati al rover.\\
Tramite l'utilizzo di un PID è possibile passare dall'errore di velocità ad un valore di uscita in percentuale (\%), che ci permette di controllare la velocità del rover. Per settare i gudagni del PID è stato usata l'applicazione PID Tuner disponibile su MATLAB. Infatti, tramite l'Identification Toolbox di MATLAB è stato possibile definire, dati dei campioni ottenuti sperimentalmente dal rover, la risposta del sistema ad un gradino.\\
Lo yaw desiderato è viene calcolato tramite
\begin{equation}
\omega_{des}(t+1)=\frac{\theta_{des}(t+1)-\theta(t)}{\Delta{T}} \ \ \
\Rightarrow \ \ \ \theta_{des}(t+1)=\theta(t)+\omega_{des}(t)\Delta{T}.
\end{equation} 
Una volta ottenuto tale valore, si calcola $ \theta_e = \theta(t) - \theta_{des} $ e tale errore viene dato in pasto ad un controllore P la cui uscita è l'angolo di sterzo del rover.
I guadagni dei due controllori sono riportati in \autoref{tab:guadPID}.
\begin{table} [H]
    \centering
    \begin{tabular}{|ccc|}
    \hline
        Guadagno & Valore controllore $v$ & Valore controllore $\omega$ \\  \hline
        $k_p$ & 0.8  & 10\\  \hline
        $k_i$ & 0.85 & 0 \\  \hline
        $k_d$ & 1    & 0 \\  \hline
    \end{tabular}
    \caption{Tabella riassuntiva dei guadagni}
    \label{tab:guadPID}    
\end{table}





\subsubsection{Filtro complementare}
Per ottenere un segnale di velocità longitudinale del rover è stato implementato un filtro complementare \cite{Filtrocompelmentare}. Questo sfrutta i dati della posizione stimata del rover e dal'accellerometro della ZED filtrandoli tramite l'utilizzo di un filtro passa basso e di un filtro passa alto. Più precisamente l'idea è quella di derivare i dati della posizione del rover per ottenere una stima della velocità che viene poi mandata al filtro passa basso, mentre per quanto riguarda i dati dell'accelerometro questi vengono integrati per ottenere sempre una stima della velocità che viene poi filtrata tramite filtro passa alto.
L'uscita è data dalla somma dei due segnali ottenuti dai filtri.
Provenendo da sensori diversi, le misurazioni della velocità di entrambi i rami presentano diverse perturbazioni del rumore.
\begin{figure} [H]
    \centering
    \includesvg[width=0.8\linewidth]{img/Filtro complementare}
    \caption{Filtro complementare}
    \label{fig:Filtro complementare}
\end{figure} 
\noindent
Il filtro è stato implementato con la seguente formula
\begin{equation}
\boldsymbol{v_{est}}=\alpha \boldsymbol{v_{pose}}+(1-\alpha) \boldsymbol{v_{acc}}
\end{equation} 
in cui il paramemtro $\alpha$ indica la percentuale delle due grandezze di ingresso che concorrono a determinare la stima della velocità.
