%La stima della posizione dei markers è stata possibile grazie ad ArUco, una libreria, che permette di trovare una corrispondenza tra i punti in un ambiente reale e la loro proiezione in un immagine 2d.  La stima della posa è di grande importanza in molte applicazioni di visione artificiale: navigazione robotica, realtà aumentata e molte altre. Questo processo si basa sulla ricerca di corrispondenze tra punti nell'ambiente reale e la loro proiezione di immagini 2D. Questo di solito è un passaggio difficile, quindi è comune utilizzare marcatori sintetici o fiduciari per renderlo più semplice. Uno degli approcci più popolari è l’uso di indicatori fiduciali quadrati binari. Il vantaggio principale di questi marcatori è che un singolo marcatore fornisce corrispondenze sufficienti (i suoi quattro angoli) per ottenere la posa della fotocamera. Inoltre, la codifica binaria interna li rende particolarmente robusti, consentendo la possibilità di applicare tecniche di rilevamento e correzione degli errori. 


