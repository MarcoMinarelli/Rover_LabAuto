L'obiettivo del progetto è far convergere il rover a rette le cui equazioni sono dinamicamente calcolate, evitando al contempo gli ostacoli inseriti nelle traiettorie percorse dal rover.  Le rette sono definite a  partire dai marker ArUco i quali vengono riconosciuti tramite tecniche di visione artificiale, mentre per la parte di obstacle avoidance sono stati implementati algoritmi che fanno uso di potenziali artificiali (Artificial Potential Field). In particolare sono stati implementati un potenziale attrattivo, che porta a far convergere il sistema al goal prefissato e uno repulsivo, che tiene lontano il rover dagli ostacoli.\\ \\
La relazione è stata suddivisa in: "Descrizione tecnica del veicolo" (\autoref{sec:desc_tec}) dove vengono descritte tutte le componenti; "Premesse Teoriche" (\autoref{sec:prem_teo}) dove vengono illustrati i fondamenti teorici degli algoritmi utilizzati; "Implementazione" (\autoref{sec:impl}) dove viene riportato il lavoro svolto relativo alla codifica dei suddetti algoritmi; "Risultati" (\autoref{sec:res}) dove vengono tracciati i grafici delle traiettorie ottenute. Infine nella sezione "Conclusioni e sviluppi futuri" (\autoref{sec:concl}) vengono proposte possibili migliorie al sistema e viene valutato il raggiungimento degli obiettivi.
