\label{sec:res}
Le prove svolte, atte a testare il corretto funzionemaento del rover e delle tecniche di controllo implememntate, sono le seguenti:
\begin{enumerate}
  \item esecuzione di una traiettoria rettangolare 
  \item esecuzione di una greca 
  \item esecuzione di una traiettoria rettangolare con APF 
\end{enumerate}
 Implementando una funzione in grado di salvare in formato .csv i dati della posizione del rover, durante il suo movimento, è stato possibile realizzare un plot in cui si mostra la traiettoria desiderata (Ground trought) a confronto con quella realizzata dal rover (Robot trajectory).

\subsection{Traiettoria rettangolare}
Nella realizzazione di tale prova sono stati eseguti due giri del rettangolo.
\begin{figure} [H]
    \centering
    \includesvg[width=\linewidth]{img/Rectangular}
    \caption{Traiettorie rettangolari}
    \label{fig:rect}
\end{figure} 

\subsection{Traiettoria Greca}
Nel grafico sono state riportate le tre traiettorie effettuarte dal rover in tre prove differenti. 
\begin{figure} [H]
    \centering
    \includesvg[width=0.8\linewidth]{img/Greca}
    \caption{Traiettoria Greca}
    \label{fig:greca}
\end{figure} 


\subsection{Traiettoria rettangolare con APF}
Nel grafico sono state riportate le due traiettorie effettuarte dal rover in due prove differenti. 
\begin{figure} [H]
    \centering
    \includesvg[width=0.8\linewidth]{img/ObsAvoidance}
    \caption{Traiettoria rettangolare con APF}
    \label{fig:rectAPF}
\end{figure} 
